
% 本模板适用于武汉大学物理学院硕博学位论文,在github上的武大毕业论文模板(https://github.com/whutug/whu-thesis,在此,向作者WHU TeX User Group致谢)的基础上做了大量的修改,以符合最新的物院毕业论文模板要求(几乎全部),如果后续物院模板更新,则可联系作者做相应的修改。-- Author:Zhengxin Tang
 


%--------------------------------- tutorial----------------------------
% 以下仅列举了部分可能用到的设置选项,更多用法请参考文档《whuthesis.pdf》
% 论文具体内容的要求可以按照《武汉大学物理学院毕业论文格式3》

%----------------------------导言区开始分割线--------------------------------

\documentclass[type = doctor,class = academic]{whu-thesis}
% type: 可选项为 bachelor, master, doctor
% class: 可选项为 academic, professional
% showframe: 显示页面布局框架
% \PassOptionsToPackage{gbnamefmt = lowercase}{biblatex} % 英文作者姓名不强制大写


% 设置
\whusetup{
  info = {
    title      = {中文论文标题}, % 标题,可使用 \\ 手动换行
    title*     = {English title},
    department  = {物理科学与技术学院},
    department* = {School of Physics and Technology, Wuhan University},
    author     = {张三},
    author*    = {San ZHANG},
    %student-id = {2022202020059},
    supervisor  = {李四},
    supervisor* = {Si LI},
    academic-title  = {教授},
    academic-title* = {Prof},
    %supervisor-outer = {}, % 校外导师(非必填)
    %supervisor-outer* = {}
    %academic-title-outer = {教授}, % 校外导师职称(非必填)
    subject = {}, % 学科名称(非必填)
    major   = {理论物理},
    major*  = {Theoretical Physics},
    research-area  = {弦论},
    research-area* = {String Theory},
    year = 2025,
    month = 8,
    keywords = {{关键词1};{关键词2};{关键词3};},
    keywords* = { {keyword1},{keyword2},{keyword3},},
    clc = O159, % 分类号
    udc = xxx,%此处为学号,懒得改函数名了O^O
  },
  style = {
    % 字体相关选项
    font = times, % 西文字体,可选项为 default, times, xits, termes
    math-font = termes, % 数学字体,可选项为 default, xits, termes
    cjk-font = fandol, % 中文字体,可选项为 windows, mac, fandol(Linux/Overleaf/TexPage), sourcehan, none
    %cjk-fakefont = true, % 使用伪粗体与伪斜体
    % 参考文献及引用相关选项
    bib-backend = bibtex, % 参考文献引擎,可选项为 bibtex, biblatex
    bib-style = numerical, % 参考文献样式,可选项为 numerical, author-year
    %cite-style = <>, % 引用样式(自定义)
    bib-resource = {ref/refs.bib}, % 参考文献数据源
    % 页面相关选项
    chapter-page-header = true, % 章节首页是否有页眉
    % bachelor-encover = true, % 本科毕业论文英文封面
    library, % 图书馆模式(去掉论文中所有的空白页)
    license, % 使用授权协议书  
    % fullwidth-stop = true, % 句号样式
    % footnote-style = <>, % 脚注编号样式
    % abstract-keywords-type  = blankline, % 摘要与关键词之间样式,可选项为 blankline, newline, vfill
    % abstract-keywords-type* = blankline, % 摘要与关键词之间样式,可选项为 blankline, newline, vfill
  }
}
\whumodule{algorithm2e}

% =============================== 新命令======================================
% 在此设置所需的命令

% 修正:使用期刊全称
\newcommand{\apj}{The Astrophysical Journal}
\newcommand{\apjl}{The Astrophysical Journal Letters}  
\newcommand{\apjs}{The Astrophysical Journal Supplement Series}
\newcommand{\aj}{The Astronomical Journal}
\newcommand{\mnras}{Monthly Notices of the Royal Astronomical Society}
\newcommand{\aap}{Astronomy \& Astrophysics}
\newcommand{\araa}{Annual Review of Astronomy and Astrophysics}
\newcommand{\pasp}{Publications of the Astronomical Society of the Pacific}
\newcommand{\nat}{Nature}
\newcommand{\science}{Science}
\newcommand{\farcs}{\mbox{$.\!\!^{\prime\prime}$}} %arcsec






%--------------------------------导言区结束分割线--------------------------

% ==================================正文==================================
\begin{document}

\tableofcontents % 目录
% \listoffigures % 图目录
% \listoftables % 表目录

% 符号表
% \begin{notation}
%   $\omega_n$ & $n$-维欧氏空间中单位球的表面积 \\
%   $\alpha_n$ & $n$-维欧氏空间中单位球的体积 \\
% \end{notation}

\mainmatter
%在src文件夹中创建章节tex,在这里include
\chapter{绪论}
\section{xxxx}
文献引用\cite{Kormendy2004,Sandage2005,mo2010}。

图 \ref{fig:hubble_sequence}
\begin{figure}[htbp]
\centering
\includegraphics[width=0.45\textwidth]{Hubble_Sequence.jpg}
\caption{: \emph{Hubble Sequence示意图。图片来源:NASA\&ESA 。}}
\label{fig:hubble_sequence}
\end{figure}

表\ref{tab:simulation_parameters}:
\begin{table}[htbp]
\centering
\caption{模拟星系参数设置}
\label{tab:simulation_parameters}
\begin{tabular}{lc}
\hline
\hline
\textbf{参数} & \textbf{取值范围} \\
\hline
总星等 ($m_{\text{total}}$) & 20-28 AB mag \\
B/T比值 & 10\%-90\% \\
\hline
\hline
\end{tabular}
\end{table}

公式\ref{eq1}
\begin{equation}
I(R) = I_e \exp\left\{-k\left[\left(\frac{R}{R_{\text{eff}}}\right)^{1/n} - 1\right]\right\}
\label{eq1}
\end{equation}


\subsection{xxxxxx}

\subsection{xxxxxx}


\include{src/chapter2.tex}
\chapter{xxxx}

\section{xxxx}
\subsection{xxxx}
\include{src/chapter4.tex}
\include{src/chapter5.tex}
\include{src/chapter6.tex}


% 当然你也可以直接在这里写,不过这样不太方便管理
%\chapter{BBBB}


% 参考文献
\printbibliography

% 致谢
\begin{acknowledgements}
  对在课题研究及论文写作过程中给予指导和帮助的导师、校内外专家、实验技术人员、同学等表示感谢。
  
在致谢时建议具体,不同的人如何助力完成你的论文,都需要特别注明。如导师、其他老师或实验技术人员、以及同学对你论文的贡献是不一样的,有指引课题方向、修改论文,也有具体教会实验操作,也有协助你做了哪方向的实验,或者给你精神安慰、陪你度过紧张的研究生生涯。

越具体越能表达你真实的感受,否则就是毫无意义的套话。

对本研究开展得到经费支持的课题项目进行致谢,如国家自然科学基金、重点研发计划等。


\end{acknowledgements}



%附录
\appendix
\ctexset{
  chapter/number = \arabic{chapter}
}
\chapter{答辩委员会决议}



\chapter{攻读硕士学位期间取得的研究成果}%此处修改硕士/博士
 \begin{enumerate}[label={[\arabic*]}]
  \item 
  \end{enumerate}


\end{document}